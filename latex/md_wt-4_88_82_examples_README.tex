The examples that come with the library will by default be linked against the built-\/in httpd. In any case, we recommend this way of exploring the examples and trying them out, even if you\textquotesingle{}ll end up deploying using Fast\+C\+GI or I\+S\+A\+PI.

You typically need the following commands to run an example ({\ttfamily foobar})\+: \begin{DoxyVerb}cd foobar # source directory for example foobar
../../build/examples/foobar/foobar.wt --docroot . --http-address 0.0.0.0 --http-port 8080 --resources-dir=../../resources
\end{DoxyVerb}


By running the examples from within their source directory, in this way the examples will find the auxiliary files in the expected places.

Some examples may need additional command line arguments, which are detailed in the R\+E\+A\+D\+M\+E.\+md for each example.

Some examples need third-\/party Java\+Script libraries (Ext\+JS or Tiny\+M\+CE).


\begin{DoxyItemize}
\item Download \href{http://yogurtearl.com/ext-2.0.2.zip}{\texttt{ Ext\+JS}}, and install it according to \href{http://www.webtoolkit.eu/wt/doc/reference/html/group__ext.html}{\texttt{ these instructions}}
\item Download \href{http://tinymce.moxiecode.com/}{\texttt{ Tiny\+M\+CE}} and install its tiny\+\_\+mce folder into the resources/ folder.
\end{DoxyItemize}

You will notice 404 File not Found errors for {\ttfamily ext/} or {\ttfamily resources/tiny\+\_\+mce/} if you are missing these Java\+Script libraries. 